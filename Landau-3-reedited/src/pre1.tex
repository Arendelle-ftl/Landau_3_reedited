\chapter{FROM THE PREFACE TO THE FIRST ENGLISH EDITION}
THE present book is one of the series on Theoretical Physics, in which we endeavour to give an up-to-date account of various departments of that science. The complete series will contain the following nine volumes:
\begin{enumerate}
	\item \textit{Mechanics}.\label{1}
	\item The classical theory of fields.\label{2}
	\item \textit{Quantum mechanics (non-relativistic theory)}.\label{3}
	\item \textit{Relativistic quantum theory}.\label{4}
	\item \textit{Statistical physics}.\label{5}
	\item \textit{Fluid mechanics}.\label{6}
	\item \textit{Theory of elasticity}.\label{7}
	\item \textit{Electrodynamics of continuous media}.\label{8}
	\item \textit{Physical kinetics}.\label{9}
\end{enumerate}

Of these, volumes \ref{4} and \ref{9} remain to be written.

The scope of modern theoretical physics is very wide, and we have, of course, made no attempt to discuss in these books all that is now included in the subject. One of the principles which guided our choice of material was not to deal with those topics which could not properly be expounded without at the same time giving a detailed account of the existing experimental results. For this reason the greater part of nuclear physics, for example, lies outside the scope of these books. Another principle of selection was not to discuss very complicated applications of the theory. Both these criteria are, of course, to some extent subjective.

We have tried to deal as fully as possible with those topics that are included. For this reason we do not, as a rule, give references to the original papers, but simply name their authors. We give bibliographical references only to work which contains matters not fully expounded by us, which by their complexity lie “on the borderline” as regards selection or rejection. We have tried also to indicate sources of material which might be of use for reference. Even with these limitations, however, the bibliography given makes no pretence of being exhaustive.

We attempt to discuss general topics in such a way that the physical significance of the theory is exhibited as clearly as possible, and then to build up the mathematical formalism. In doing so, we do not aim at “mathematical rigour” of exposition, which in theoretical physics often amounts to self-deception.

The present volume is devoted to non-relativistic quantum mechanics. By “relativistic theory” we here mean, in the widest sense, the theory of all quantum phenomena which significantly depend on the velocity of light. The volume on this subject (volume \ref{4}) will therefore contain not only Dirac’s relativistic theory and what is now known as quantum electrodynamics, but also the whole of the quantum theory of radiation.


\vspace{4ex}
\noindent L.D. LANDAU and E.M. LIFSHITZ, Institute of Physical Problems, USSR Academy of Sciences



\noindent\date{August 1956}