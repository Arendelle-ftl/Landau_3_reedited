\chapter{NOTATION}
\begin{itemize}
	\item Operators are denoted by a circumflex: $\hat{f}$
	\item $ \d V $ volume element in coordinate space
	\item $ \d q $ element in configuration space
	\item $ \d^3 p $ element in momentum space
	\item $ f_{nm} = f_{mn} = \langle n|f|m\rangle $ matrix elements of the quantity $ f $ (see definition in \S\ref{Matrices})
	\item $ \omega_{nm} = (E_n − E_m)/\h $ transition frequency
	\item $ \left\{\hat{f},\hat{g} \right\}=\hat{f}\hat{g}-\hat{g}\hat{f} $ commutator of two operators
	\item $\hat{H}$ Hamiltonian
	\item $\delta_l$ phase shifts of wave functions
	\item Atomic and Coulomb units (see beginning of \S\ref{Motion in a Coulomb field (spherical polar coordinates)})
	\item Vector and tensor indices are denoted by Latin letters $ i, k, l $
	\item $ e_{ikl} $ antisymmetric unit tensor (see \S\ref{Angular momentum})
	\item References to other volumes in the Course of Theoretical Phyncs:
	\begin{itemize}
		\item \textit{Mechanics} $ = $ Vol. \ref{1} (\textit{Mechanics}, third English edition, 1976).
		\item \textit{Fields} $ = $ Vol. \ref{2} (\textit{The Classical Theory of Fields}, fourth English edition, 1975).
		\item \textit{RQT} or \textit{Relativistic Quantum Theory} $ = $ Vol. \ref{4} (\textit{Relativistic Quantum Theory}, first English edition, Part 1, 1971; Part 2, 1974); the second English edition appeared in one volume as \textit{Quantum Electrodynamics}, 1982.
	\end{itemize}
All are published by Pergamon Press.
\end{itemize}



