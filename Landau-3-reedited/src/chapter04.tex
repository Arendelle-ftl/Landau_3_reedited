\chapter{ANGULAR MOMENTUM}
\section{Angular momentum}\label{Angular momentum}
IN \S\ref{Momentum}, to derive the law of conservation of momentum, we have made use of the homogeneity of space relative to a closed system of particles. Besides its homogeneity, space has also the property of isotropy: all directions in it are equivalent. Hence the Hamiltonian of a closed system cannot change when the system rotates as a whole through an arbitrary angle about an arbitrary axis. It is sufficient to require the fulfilment of this condition for an infinitely small rotation.


Let $\delta\bm{\phi}$ be the vector of an infinitely small rotation, equal in magnitude to the angle $\delta\phi$ of the rotation and directed along the axis about which the rotation takes place. The changes $\delta\bm{r}_a$ (in the radius vectors $ \bm{r}_a $ of the particles) in such a rotation are
\[ \delta\bm{r}_a=\delta\bm{\phi}\times\bm{r}_a. \]



An arbitrary function $ \psi(\bm{r}_1,\bm{r}_2,\dots) $ is thereby transformed into the function
\begin{multline*}
\psi(\bm{r}_1+\delta\bm{r}_1,\bm{r}_2+\delta\bm{r}_2,\dots)=\psi(\bm{r}_1+\bm{r}_2+\dots)+\sum_{a}\delta\bm{r}_a\cdot\nabla_a\psi=\\
=\psi(\bm{r}_1+\bm{r}_2+\dots)+\sum_{a}\delta\bm{\phi}\times\bm{r}_a\cdot\nabla_a\psi=\\
=\left(1+\delta\bm{\phi}\cdot\sum_{a}\bm{r}_a\times\nabla_a \right)\psi(\bm{r}_1+\bm{r}_2+\dots).
\end{multline*}
The expression
\[ 1+\delta\bm{\phi}\cdot\sum_{a}\bm{r}_a\times\nabla_a \]
is the operator of an infinitely small rotation. The fact that an infinitely small rotation does not alter the Hamiltonian of the system is expressed (cf. \S\ref{Momentum}) by the commutability of the “rotation operator” with the operator $ \hat{H} $. Since $\delta\bm{\phi}$ is a constant vector, this condition reduces to the relation
\begin{equation}\label{26.1}
\left(\sum_{a}\bm{r}_a\times\nabla_a\right)\hat{H}-\hat{H}\left(\sum_{a}\bm{r}_a\times\nabla_a\right)=0,
\end{equation}
which expresses a certain law of conservation.

The quantity whose conservation for a closed system follows from the property of isotropy of space is the \textit{angular momentum} of the system (cf. \textit{Mechanics}, \S9). Thus the operator $ \sum\bm{r}_a\times\nabla_a $ must correspond exactly, apart from a constant factor, to the total angular momentum of the system, and each of the terms $ \bm{r}_a\times\nabla_a $ of this sum corresponds to the angular momentum of an individual particle.

The coefficient of proportionality must be put equal to $ -\i\h $; then the expression for the angular momentum operator of a particle is $ -\i\h\bm{r}\times\nabla=\bm{r}\times\hat{\bm{p}} $ and corresponds exactly to the classical expression $ \bm{r}\times\bm{p} $. Henceforward we shall always use the angular momentum measured in units of $ \h $. The angular momentum operator of a particle, so defined, will be denoted by $ \hat{\bm{l}} $, and that of the whole system by $ \hat{\bm{L}} $. Thus the angular momentum operator of a particle is
\begin{equation}\label{26.2}
\h\hat{\bm{l}}=\bm{r}\times\hat{\bm{p}}=-\i\h\bm{r}\times\nabla
\end{equation}
or, in components,
\[ \h \hat{l_x}=y\hat{p}_z-z\hat{p}_y,\quad\hat{l_y}=z\hat{p}_x-x\hat{p}_z,\quad\hat{l_z}=x\hat{p}_y-y\hat{p}_x. \]


For a system which is in an external field, the angular momentum is in general not conserved. However, it may still be conserved if the field has a certain symmetry. Thus, if the system is in a centrally symmetric field, all directions in space at the centre are equivalent, and hence the angular momentum about this centre will be conserved. Similarly, in an axially symmetric field, the component of angular momentum along the axis of symmetry is conserved. All these conservation laws holding in classical mechanics are valid in quantum mechanics also.

In a system where angular momentum is not conserved, it does not have definite values in the stationary states. In such cases the mean value of the angular momentum in a given stationary state is sometimes of interest. It is easily seen that, in any non-degenerate stationary state, the mean value of the angular momentum is zero. For, when the sign of the time is changed, the energy does not alter, and, since only one stationary state corresponds to a given energy level, it follows that when $ t $ is changed into $ -t $ the state of the system must remain the same. This means that the mean values of all quantities, and in particular that of the angular momentum, must remain unchanged. But when the sign of the time is changed, so is that of the angular momentum, and we have $ \bar{\bm{L}}=-\bar{\bm{L}} $ , whence it follows that $ \bar{\bm{L}}=0 $. The same result can be obtained by starting from the mathematical definition of the mean value as being the integral of $\psi*\hat{\bm{L}}\psi$. The wave functions of non-degenerate states are real (see the end of \S\ref{The fundamental properties of Schr\"odinger's equation}). Hence the expression
\[ \bar{\bm{L}}=-\i\h\int\psi^*\left(\sum_{a}\bm{r}_a\times\nabla_a \right)\psi\d q \]
is purely imaginary, and since $ \bar{\bm{L}} $ must, of course, be real, it is evident that $ \bar{\bm{L}}=0 $.

Let us derive the rules for commutation of the angular momentum operators with those of coordinates and linear momenta. By means of the relations \eqref{16.2} we easily find
\begin{equation}\label{26.3}
\begin{split}
\{\hat{l}_x,x\}=0,\quad\{\hat{l}_x,y \}&=\i z,\quad\{\hat{l_x},z  \}=-\i y,\\
\{\hat{l}_y,y \}=0,\quad\{\hat{l}_y,z \}&=\i x,\quad\{\hat{l}_y,x  \}=-\i z,\\
\{\hat{l}_z,z \}=0,\quad\{\hat{l}_z,x \}&=\i y,\quad\{\hat{l}_z,y \}=-\i x.
\end{split}
\end{equation}
For instance,
\[ \hat{l}_xy-y\hat{l}_x=\frac{1}{\h}(y\hat{p}_z-z\hat{p_y})y-y(y\hat{p}_z-z\hat{p_y})\frac{1}{\h}=-\frac{z}{\h}{\hat{p}_y,y}=\i z \]




All the relations \eqref{26.3} can be written in tensor form as follows:
\begin{equation}\label{26.4}
\{\hat{l}_i,x_k\}=\i e_{ikl}x_l
\end{equation}
where $ e_{ikl} $ is the \textit{antisymmetric unit tensor} of rank three,\footnote{The antisymmetric unit tensor of rank three, $ e_{ikl} $ (also called the unit axial tensor), is defined as a tensor antisymmetric in all three suffixes, with $ e_{123} =1  $. It is evident that, of its $ 27 $ components, only $ 6 $ are not zero, namely those in which the suffixes $ i, k, l $ form some permutation of $ 1, 2, 3 $. Such a component is +1 if the permutation $ i, k, l $ is obtained from $ 1, 2, 3 $ by an even number of transpositions of pairs of figures, and is $ -1 $ if the number of transpositions is odd. Clearly 
\[ e_{ikl}e_{ikm} = 2\delta_{lm}, \quad e_{ikl}e_{ikl}= 6. \]	
The components of the vector $ \bm{C} = \bm{A} \times \bm{B} $ which is the vector product of the two vectors $ \bm{A} $ and $ \bm{B} $ can be written by means of the tensor $ e_{ikl} $ in the form
\[ C_i=e_{ikl}A_kB_l. \]
}and summation is implied over those suffixes which appear twice (called \textit{dummy suffixes}).

It is easily seen that a similar commutation rule holds for the angular momentum and linear momentum operators:
\begin{equation}\label{26.5}
\{\hat{l}_i,\hat{p}_k \}=\i e_{ikl}\hat{p}_l.
\end{equation}



By means of these formulae, it is easy to find the rules for commutation of the operators with one another. We have
\begin{multline*}
\h(\hat{l}_x\hat{l}_y-\hat{l}_y\hat{l}_x)=\hat{l}_x(z\hat{p}_x-x\hat{p}_z)-(z\hat{p}_x-x\hat{p}_z)\hat{l}_x=\\
=(\hat{l}_xz-z\hat{l}_x)\hat{p}_x-x(\hat{l}_x\hat{p}_z-\hat{p}_z\hat{l}_x)=\i y\hat{p}_x+\i x\hat{p}_y=\i\h\hat{l}_z.
\end{multline*}
Thus
\begin{equation}\label{26.6}
\{\hat{l}_y,\hat{l}_z  \}=\i\hat{l}_x,\quad\{\hat{l}_z,\hat{l}_x  \}=\i\hat{l}_y,\quad\{\hat{l}_x,\hat{l}_y  \}=\i\hat{l}_z,
\end{equation}
or
\begin{equation}\label{26.7}
\{\hat{l}_i,\hat{l}_k  \}=\i e_{ikl}\hat{l}_l
\end{equation}
Exactly the same relations hold for the operators of the total angular momentum of the system. For, since the angular momentum operators of different individual particles commute, we have, for instance,
\[ \sum_{a}\hat{l}_{ay}\sum_{a}\hat{l}_{az}-\sum_{a}\hat{l}_{az}\sum_{a}\hat{l}_{ay}=\sum_{a}\left(\hat{l}_{ay}\hat{l}_{az}-\hat{l}_{az}\hat{l}_{ay}  \right). \]
Thus
\begin{equation}\label{26.8}
\{\hat{L}_y,\hat{L}_z \}=\i\hat{L}_x,\quad\{\hat{L}_z,\hat{L}_x \}=\i\hat{L}_y,\quad\{\hat{L}_x,\hat{L}_y \}=\i\hat{L}_z.
\end{equation}
The relations \eqref{26.8} show that the three components of the angular momentum cannot simultaneously have definite values (except in the case where all three components simultaneously vanish: see below). In this respect the angular momentum is fundamentally different from the linear momentum, whose three components are simultaneously measurable.

From the operators $ \hat{L}_x,\hat{L}_y,\hat{L}_z $ we can form the operator of the square of the modulus of the angular momentum vector, and which we denote by :
\begin{equation}\label{26.9}
\hat{\bm{L}}^2=\hat{L}_x^2+\hat{L}_y^2+\hat{L}_z^2.
\end{equation}
This operator commutes with each of the operators :
\begin{equation}\label{26.10}
\{\hat{\bm{L}}^2,\hat{L}_x \}=0,\quad\{\hat{\bm{L}}^2,\hat{L}_y \}=0,\quad\{\hat{\bm{L}}^2,\hat{L}_z \}=0.
\end{equation}
Using \eqref{26.8}, we have
\[ \{\hat{L}_x^2,\hat{L}_z  \}=\hat{L}_x\{\hat{L}_x,\hat{L}_z \}+\{\hat{L}_x,\hat{L}_z \}\hat{L}_x=-\i\left(\hat{L}_x\hat{L}_y+\hat{L}_y\hat{L}_x \right) \]
\[ \{\hat{L}_y^2,\hat{L}_z  \}=\i\left(\hat{L}_x\hat{L}_y+\hat{L}_y\hat{L}_x \right)  \]
\[ \{\hat{L}_z^2,\hat{L}_z \}=0 \]
Adding these equations, we have $ \{\hat{\bm{L}}^2,\hat{L}_z  \}=0 $. Physically, the relations \eqref{26.10} mean that the square of the angular momentum, i.e. its modulus, can have a definite value at the same time as one of its components.

Instead of the operators $ \hat{L}_x,\hat{L}_y $ it is often more convenient to use the complex combinations
\begin{equation}\label{26.11}
\hat{L}_{+}=\hat{L}_x+\i\hat{L}_y,\quad\hat{L}_{-}=\hat{L}_x-\i\hat{L}_y
\end{equation}
It is easily verified by direct calculation using \eqref{26.8} that the following commutation rules hold:
\begin{equation}\label{26.12}
\{\hat{L}_{+},\hat{L}_{-}  \}=2\hat{L}_z,\quad\{\hat{L}_z,\hat{L}_{+} \}=\hat{L}_{+},\quad\{ 
\hat{L}_z,\hat{L}_{-} \}=-\hat{L}_{-},
\end{equation}
and it is also not difficult to see that
\begin{equation}\label{26.13}
\hat{\bm{L}}^2=\hat{L}_{+}\hat{L}_{-}+\hat{L}_z^2-\hat{L}_z=\hat{L}_{-}\hat{L}_{+}+\hat{L}_z^2+\hat{L}_z.
\end{equation}
Finally, we shall give some frequently used expressions for the angular momentum operator of a single particle in spherical polar coordinates. Defining the latter by means of the usual relations
\[ x=r\sin\theta\cos\phi,\quad y=r\sin\theta\sin\phi,\quad z=r\cos\theta, \]
we have after a simple calculation
\begin{equation}\label{26.14}
\hat{l}_z=-\i\frac{\p}{\p \phi},
\end{equation}
\begin{equation}\label{26.15}
\hat{l}_{\pm}=\e^{\pm\i\phi}\left(\pm\frac{\p}{\p\theta}+\i\cot\theta\frac{\p}{\p\phi} \right).
\end{equation}
Substitution in \eqref{26.13} gives the squared angular momentum operator of the particle:
\begin{equation}\label{26.16}
\hat{\bm{l}}^2=-\left[\frac{1}{\sin^2\theta}\frac{\p^2}{{\p\phi}^2}+\frac{1}{\sin\theta}\frac{\p}{\p\theta}\left(\sin\theta\frac{\p}{\p\phi} \right) \right]
\end{equation}
It should be noticed that this is, apart from a factor, the angular part of the Laplacian operator.





\section{Eigenvalues of the angular momentum}\label{Eigenvalues of the angular momentum}


In order to determine the eigenvalues of the component, in some direction, of the angular momentum of a particle, it is convenient to use the expression for its operator in spherical polar coordinates, taking the direction in question as the polar axis. According to formula \eqref{26.14}, the equation $ \hat{l}_z\psi=l_z\psi $ can be written in the form
\begin{equation}\label{27.1}
-\i\frac{\p\psi}{\p\phi}=l_z\psi.
\end{equation}
Its solution is
\[ \psi=f(r,\theta)\e^{\i l_z\phi}, \]
where $ f (r, \theta) $ is an arbitrary function of $ r $ and $ \theta $. If the function $\psi$ is to be single-valued, it must be periodic in $\phi$, with period $ 2\pi $. Hence we find\footnote{The customary notation for the eigenvalues of the angular momentum component is $ m $, which also denotes the mass of a particle, but this should not lead to any confusion.}
\begin{equation}\label{27.2}
l_z=m,\quad m=0,\pm 1,\pm 2,\dots
\end{equation}
Thus the eigenvalues $ l_z $ are the positive and negative integers, including zero. The factor depending on $\phi$, which characterizes the eigenfunctions of the operator $ \hat{l}_z $, is denoted by
\begin{equation}\label{27.3}
\Psi_m(\phi)=\frac{1}{\sqrt{2}}\e^{\i m\phi}.
\end{equation}
These functions are normalized so that
\begin{equation}\label{27.4}
\int_{0}^{2\pi}\Psi_m^*(\phi)\Psi_{m'}(\phi)\d\phi=\delta_{mm'}.
\end{equation}
The eigenvalues of the $ z $-component of the total angular momentum of the system are evidently also equal to the positive and negative integers:
\begin{equation}\label{27.5}
L_z=M,\quad M=0,\pm 1,\pm 2,\dots
\end{equation}
(this follows at once from the fact that the operator $ \hat{L}_z $ is equal to the sum of the commuting operators $ \hat{l}_z $ for the individual particles).

Since the direction of the $ z $-axis is in no way distinctive, it is clear that the same result is obtained for $ \hat{L}_x,\hat{L}_y $ and in general for the component of the angular momentum in any direction: they can all take integral values only. At first sight this result may appear paradoxical, particularly if we apply it to two directions infinitely close to each other. In fact, however, it must be remembered that the only common eigenfunction of the operators $ \hat{L}_x,\hat{L}_y,\hat{L}_z $ corresponds to the simultaneous values
\[ {L}_x={L}_y={L}_z=0; \]
in this case the angular momentum vector is zero, and consequently so is its projection upon any direction. If even one of the eigenvalues $ {L}_x,{L}_y,{L}_z $ is not zero, the operators $ \hat{L}_x,\hat{L}_y,\hat{L}_z $ have no common eigenfunctions. In other words, there is no state in which two or three of the angular momentum components in different directions simultaneously have definite values different from zero, so that we can say only that one of them is integral.

The stationary states of a system which differ only in the value of $ M $ have the same energy; this follows from general considerations, based on the fact that the direction of the $ z $-axis is in no way distinctive. Thus the energy levels of a system whose angular momentum is conserved (and is not zero) are always degenerate.\footnote{This is a particular case of the general theorem, mentioned in \S\ref{Stationary states}, which states that the levels are degenerate when two or more conserved quantities exist whose operators do not commute. Here the components of the angular momentum are such quantities.
}

Let us now look for the eigenvalues of the square of the angular momentum. We shall show how these values may be found, starting from the commutation rules \eqref{26.8} only. We denote by $\psi_M$ the wave functions of the stationary states with the same value of $ \bm{L}^2 $, belonging to one degenerate energy level, and distinguished by the value of $ M $.\footnote{Here it is supposed that there is no additional degeneracy leading to the same value of the energy for different values of the squared angular momentum. This is true for a discrete spectrum (except for the case of what is called \textit{accidental degeneracy} in a Coulomb field; see §36) and in general untrue for the energy levels of a continuous spectrum. However, even when such additional degeneracy is present, we can always choose the eigenfunctions so that they correspond to states with definite values of $ \bm{L}^2 $, and then we can choose from these the states with the same values of $ E $ and $ \hat{\bm{L}}^2 $. This is mathematically expressed by the fact that the matrices of commuting operators can always be simultaneously brought into diagonal form. In what follows we shall, in such cases, speak, for the sake of brevity, as if there were no additional degeneracy, bearing in mind that the results obtained do not in fact depend on this assumption, by what we have just said.}

First of all we note that, since the two directions of the $ z $-axis are physically equivalent, for every possible positive value $ M = |M| $ there is a corresponding negative value $ M = −|M| $. Let $ L $ (a positive integer or zero) denote the greatest possible value of $ |M| $ for a given $ \hat{\bm{L}}^2 $. The existence of this upper limit follows because is the operator $ \hat{\bm{L}}^2-\hat{L}_z^2=\hat{L}_x^2+\hat{L}_y^2 $ of the essentially positive physical quantity $ L_x^2 + L_y^2 $, and its eigenvalues therefore cannot be negative.

Applying the operator $ \hat{L}_z\hat{L}_{\pm} $ to the eigenfunction $\psi_M$ of the operator $ \hat{L}_z $ and using the commutation rule \ref{26.12}, we obtain
\begin{equation}\label{27.6}
\hat{L}_z\hat{L}_{\pm}\psi_M=(M+1)\hat{L}_{\pm}\psi_M.
\end{equation}
Hence we see that the function $ \hat{L}_{\pm}\psi_M $ is (apart from a normalization constant) the eigenfunction corresponding to the value $ M \pm 1 $ of the quantity $ L_z $:
\begin{equation}\label{27.7}
\psi_{M+1}=\mathrm{const}\cdot\hat{L}_{+}\psi_M,\quad\psi_{M-1}=\mathrm{const}\cdot L_-\psi_M.
\end{equation}
If we put $ M = L $ in the first of these equations, we must have identically
\begin{equation}\label{27.8}
\hat{L}_+\psi_L=0
\end{equation}
since there is by definition no state with $ M > L $. Applying the operator $ \hat{L}_- $ to this equation and using the relation \eqref{26.13}, we obtain
\[ \hat{L}_-\hat{L}_+\psi_L=(\hat{\bm{L}}^2-\hat{L}_z^2-\hat{L}_z)\psi_L=0. \]
Since, however, the $\psi_M$ are common eigenfunctions of the operators $ \hat{\bm{L}}^2 $ and $ \hat{L}_z $ , we have
\[ \hat{\bm{L}}^2\psi_L=\bm{L}^2\psi_L,\quad\hat{L}_z^2\psi_L=L^2\psi_L,\quad\hat{L}_z\psi_L=L\psi_L, \]
so that the equation found above gives
\begin{equation}\label{27.9}
\bm{L}^2=L(L+1).
\end{equation}


Formula \eqref{27.9} determines the required eigenvalues of the square of the angular momentum; the number $ L $ takes all positive integral values, including zero. For a given value of $ L $, the component $ L_z = M $ of the angular momentum can take the values
\begin{equation}\label{27.10}
M=L,L-1,\dots,-L,
\end{equation}
i.e. $ 2L + 1 $ different values in all. The energy level corresponding to the angular momentum $ L $ thus has $ (2 L + 1) $-fold degeneracy; this is usually called degeneracy with respect to the direction of the angular momentum. A state with angular momentum $ L = 0 $ (when all three components are also zero) is not degenerate. The wave function of such a state is spherically symmetric, as is evident from the fact that the change in it under any infinitesimal rotation, given by $ \hat{\bm{L}}\psi $, is in this case zero.

We shall often, for the sake of brevity, and in accordance with custom, speak of the “angular momentum” L of a system, understanding by this an angular momentum whose square is $ L (L + 1) $; the $ z $-component of the angular momentum is usually called just the “angular momentum component”.

The angular momentum of a single particle is denoted by the small letter $ l $, for which formula \eqref{27.9} becomes
\begin{equation}\label{27.11}
\bm{l}^2=l(l+1).
\end{equation}



Let us calculate the matrix elements of the quantities $ L_x $ and $ L_y $ in a representation in which $ L_z $ and $ \bm{L}^2 $, as well as the energy, are diagonal (M. Born, W. Heisenberg and P. Jordan 1926). First of all, we note that, since the operators $\hat{L}_x$ and $ \hat{L}_y $ commute with the Hamiltonian, their matrices are diagonal with respect to the energy, i.e. all matrix elements for transitions between states of different energy (and different angular momentum $ L $) are zero. Thus it is sufficient to consider the matrix elements for transitions within a group of states with different values of $ M $, corresponding to a single degenerate energy level.

It is seen from formulae \eqref{27.7} that, in the matrices of the operators $ \hat{L}_+ $ and $ \hat{L}_- $, only those elements are different from zero which correspond to transitions $ M - 1 \to M $ and $ M \to M - 1 $ respectively. Taking this into account, we find the diagonal matrix elements on both sides of the equation \eqref{26.13}, obtaining\footnote{In the symbols for the matrix elements, we omit for brevity all suffixes with respect to which they are diagonal (including $ L $).}
\[ L(L+1)=\langle M|L_+|M-1\rangle\langle M-1|L_-|M\rangle+M^2-M. \]
Noticing that, since the operators and are Hermitian,
\[\langle M-1|L_-|M\rangle={\langle M|L_+|M-1\rangle}^*  ,\]
we can rewrite this equation in the form
\[ |\langle M|L_+|M-1\rangle|^2=L(L+1)-M(M+1)=(L-M+1)(L+M), \]
whence\footnote{The choice of sign in this formula corresponds to the choice of the phase factors in the eigenfunctions of the angular momentum.
}
\begin{equation}\label{27.12}
\langle M|L_+|M-1\rangle=\langle M-1|L_-|M\rangle=\sqrt{(L+M)(L-M+1)}.
\end{equation}
Hence we have for the non-zero matrix elements of the quantities $ L_x $ and $ L_y $ themselves
\begin{equation}\label{27.13}
\begin{split}
\langle M|L_x|M-1\rangle&=\langle M-1|L_x|M\rangle=\frac{1}{2}\sqrt{(L+M)(L-M+1)},\\
\langle M|L_y|M-1\rangle&=-\langle M-1|L_y|M\rangle=-\frac{\i}{2}\sqrt{(L+M)(L-M+1)}.
\end{split}
\end{equation}


The diagonal elements in the matrices of the quantities $ L_x $ and $ L_y $ are zero. Since a diagonal matrix element gives the mean value of the quantity in the corresponding state, it follows that the mean values $\bar{{L}_x}$ and $ \bar{{L}_y} $ are zero in states having definite values of $ L_z $. Thus, if the angular-momentum component in a given direction in space has a definite value, the vector $\bar{\bm{L}}$ itself is in that direction.





\section{Eigenfunctions of the angular momentum}\label{Eigenfunctions of the angular momentum}


The wave function of a particle is not completely determined when the values of $ l $ and $ m $ are prescribed. This is seen from the fact that the expressions for the operators of these quantities in spherical polar coordinates contain only the angles $\theta$ and $\phi$, so that their eigenfunctions can contain an arbitrary factor depending on $ r $. We shall here consider only the angular part of the wave function which characterizes the eigenfunctions of the angular momentum, and denote this by $ Y_{lm} (\theta,\phi) $, with the normalization condition:
\[ \int|Y_{lm}|^2\d o=1, \]
where $ \d o = \sin\theta\d\theta\d\phi $ is an element of solid angle.

We shall see that the problem of determining the common eigenfunctions of the operators $\hat{\bm{l}}^2$ and $ \hat{l}_z $ admits of separation of the variables $\theta$ and $\phi$, and these functions can be sought in the form
\begin{equation}\label{28.1}
Y_{lm}=\Psi_m(\phi)\Theta_{lm}(\theta),
\end{equation}
where $ \Theta_m(\phi) $ are the eigenfunctions of the operator $\hat{l}_z$, which are given by formula \eqref{27.3}. Since the functions $\Theta_m$ are already normalized by the condition \eqref{27.4}, the $\Theta_{lm}$ must be normalized by the condition
\begin{equation}\label{28.2}
\int_{0}^{\pi}|\Theta_{lm}|^2\sin\theta\d\theta.
\end{equation}


The functions $ Y_{lm} $ with different $ l $ or $ m $ are automatically orthogonal:
\begin{equation}\label{28.3}
\int_{0}^{2\pi}\int_{0}^{\pi}Y_{l'm'}^*Y_{lm}\sin\theta\d\theta\d\phi=\delta_{ll'}\delta_{mm'},
\end{equation}
as being the eigenfunctions of angular momentum operators corresponding to different eigenvalues. The functions $\Psi_{m}(\phi)$ separately are themselves orthogonal (see \eqref{27.4}), as being the eigenfunctions of the operator $ \hat{l}_z $ corresponding to different eigenvalues $ m $ of this operator. The functions $\Theta_{lm}(\phi)$ are not themselves eigenfunctions of any of the angular momentum operators; they are mutually orthogonal for different $ l $, but not for different $ m $.

The most direct method of calculating the required functions is by directly solving the problem of finding the eigenfunctions of the operator $\hat{\bm{l}}^2$ written in spherical polar coordinates (formula \eqref{26.16}). The equation $ \hat{\bm{l}}^2\psi=\bm{l}^2\psi $ is:
\[ \frac{1}{\sin\theta}\frac{\p}{\p\theta}\left(\sin\theta\frac{\p\psi}{\p\theta} \right)+\frac{1}{\sin^2\theta}\frac{\p^2\psi}{{\p\phi}^2}+l(l+1)\psi=0. \]
Substituting in this equation the form \eqref{28.1} for $\psi$, we obtain for the function $\Theta_{lm}$ the equation
\begin{equation}\label{28.4}
\frac{1}{\sin\theta}\frac{\d}{\d\theta}\left(\sin\theta\frac{\d\Theta_{lm}}{\d\theta} \right)-\frac{m^2}{\sin^2\theta}\Theta_{lm}+l(l+1)\Theta_{lm}=0.
\end{equation}
This equation is well known in the theory of spherical harmonics. It has solutions satisfying the conditions of finiteness and single-valuedness for positive integral values of $ l \geqslant |m| $, in agreement with the eigenvalues of the angular momentum obtained above by the matrix method. The corresponding solutions are what are called \textit{associated Legendre polynomials} $ P_l^m(\cos\theta) $ (see \S c of the Mathematical Appendices). Using the normalization condition \eqref{28.2}, we find\footnote{The choice of the phase factor is not, of course, determined by the normalization condition. The definition (28.5) used in this book is the most natural from the viewpoint of the theory of addition of angular momenta. It differs by a factor il from the one usually adopted. The advantages of this choice will be clear from the footnotes in \S\S60, 106 and 107.}
\begin{equation}\label{28.5}
\Theta_{lm}=(-1)^m\i^l\sqrt{\frac{(2l+1)}{2}\frac{(l-m)!}{(l+m)!}}P_l^m(\cos\theta).
\end{equation}
Here it is supposed that $ m \geqslant0 $. For negative $ m $, we use the definition
\begin{equation}\label{28.6}
\Theta_{l,-|m|}=(-1)^m\Theta_{l|m|}
\end{equation}
In other words, $\Theta_{lm}$ for $ m < 0 $ is given by \eqref{28.5} with $ |m| $ instead of $ m $ and the factor $ (-1)^m $ omitted.

Thus the angular momentum eigenfunctions are mathematically just spherical harmonic functions normalized in a particular way. For reference, the complete expression embodying the above definitions is
\begin{equation}\label{28.7}
Y_{lm}(\theta,\phi)=(-1)^{(m+|m|)/2}\i^l\left[\frac{2l+1}{4\pi}\frac{(l-|m|)!}{(l+|m|)!} \right]^{1/2}P_l^{|m|}(\cos\theta)\e^{\i m\phi}.
\end{equation}
In particular,
\begin{equation}\label{28.8}
Y_{l0}=\i^l\sqrt{\frac{2l+1}{4\pi}}P_l(\cos\theta).
\end{equation}
It is evident that the functions differing in the sign of $ m $ are related by
\begin{equation}\label{28.9}
(-1)^{l-m}Y_{l,-m}=Y_{lm}^*.
\end{equation}


For $ l = 0 $ (so that $ m = 0 $ also) the spherical harmonic function reduces to a constant. In other words, the wave functions of the states of a particle with zero angular momentum depend only on $ r $, i.e. they have complete spherical symmetry, in agreement with the general statement in \S\ref{Eigenvalues of the angular momentum}.

For a given $ m $, the values of $ l $ starting from $ |m| $ denumerate the successive eigenvalues of the quantity $ \bm{l}^2 $ in order of increasing magnitude. Hence, from the general theory of the zeros of eigenfunctions (\S\ref{General properties of motion in one dimension}), we can deduce that the function $\Theta_{lm}$ becomes zero for $ l - |m| $ different values of the angle $\theta$; in other words, it has as nodal lines $ l - |m| $ “lines of latitude” on the sphere. If the complete angular functions are taken with the real factors $\cos m\phi$ or $ \sin m\phi $ instead of\footnote{Each such function corresponds to a state in which $ l_z $ does not have a definite value, but can have the values $\pm m$ with equal probability.} $ \e^{\pm\i|m|\phi} $, they have as further nodal lines $ |m| $ “lines of longitude”; the total number of nodal lines is thus $ l $.

Finally, we shall show how the functions $\Theta_{lm}$ may be calculated by the matrix method. This is done similarly to the calculation of the wave functions of an oscillator in \S\ref{The linear oscillator}. We start from the equation \eqref{27.8}:
\[ \hat{l}_+Y_{ll}=0. \]
Using the expression \eqref{26.15} for the operator $\hat{l}_+$ and substituting
\[ Y_{ll}=\frac{1}{\sqrt{2\pi}}\e^{\i l\phi}\Theta_{ll}(\theta), \]
we obtain for $\Theta_{ll}$ the equation
\[ \frac{\d\Theta_{ll}}{\d\theta}-l\cot\theta\cdot\Theta_{ll}=0, \]
whence $ \Theta_{ll} = \mathrm{const}\cdot\sin^l\theta $. Determining the constant from the normalization condition, we find
\begin{equation}\label{28.10}
\Theta_{ll}=(\i)^l\sqrt{\frac{(2l+1)!}{2}}\frac{1}{2^ll!}\sin^l\theta.
\end{equation}
Next, using \eqref{27.12}, we write
\[ \hat{l}_-Y_{l,m+1}=(l_-)_{m,m+1}Y_{lm}=\sqrt{(l-m)(l+m+1)}Y_{lm}. \]
A repeated application of this formula gives
\[ \sqrt{\frac{(l-m)!}{(l+m)!}}Y_{lm}=\frac{1}{\sqrt{(2l)!}}\hat{l}_-^{l-m}Y_{ll}. \]
The right-hand side of this equation is easily calculated by means of the expression \eqref{26.15} for the operator $\hat{l}_-$. We have
\[ \hat{l}_-\left[f(\theta)\e^{\i m\phi} \right]=\e^{\i(m-1)\phi}\sin^{1-m}\theta\frac{\d}{(\d\cos\theta)^{l-m}}\left(f\sin^m\theta \right) \]
A repeated application of this formula gives
\[ \hat{l}_-^{l-m}\e^{\i l\phi}\Theta_{ll}=\e^{\i m\phi}\sin^{-m}\theta\frac{\d^{l-m}}{(\d\cos\theta)^{l-m}}\left(\sin^l\theta\cdot\Theta_{ll} \right) .\]



Finally, using these relations and the expression \eqref{28.10} for $\Theta_{ll}$, we obtain the formula
\begin{equation}\label{28.11}
\Theta_{lm}(\theta)=(-\i)^l\sqrt{\frac{2l+1}{2}\frac{(l+m)!}{(l-m)!}}\frac{1}{2^ll!\sin^m\theta}\frac{\d^{l-m}}{(\d\cos\theta)^{l-m}}\sin^{2l}\theta,
\end{equation}
which is the same as \eqref{28.5}.





\section{Matrix elements of vectors}\label{Matrix elements of vectors}


Let us again consider a closed system of particles;\footnote{All the results in this section are valid also for a particle in a centrally symmetric field (and in general whenever the total angular momentum of the system is conserved).} let $ f $ be any scalar physical quantity characterizing the system, and $\hat{f}$ the operator corresponding to this quantity. Every scalar is invariant with respect to rotation of the coordinate system. Hence the scalar operator $\hat{f}$ does not vary when acted on by a rotation operator, i.e. it commutes with a rotation operator. We know, however, that the operator of an infinitely small rotation is the same, apart from a constant factor, as the angular momentum operator, so that
\begin{equation}\label{29.1}
\{\hat{f},\hat{\bm{L}} \}=0
\end{equation}
From the commutability of $\hat{f}$ with the angular momentum operator it follows that the matrix of $\hat{f}$ with respect to transitions between states with definite values of $ L $ and $ M $ is diagonal with respect to these suffixes. Moreover, since the specification of $ M $ defines only the orientation of the system relative to the axes of coordinates, and the value of a scalar is independent of this orientation, we can say that the matrix elements $ \langle n'LM|f|nLM\rangle $ are independent of the value of $ M $; $ n $ conventionally denotes all the quantum numbers other than $ L $ and $ M $ which define the state of the system. A formal proof of this assertion can be obtained from the commutativity of the operators $\hat{f}$ and $ \hat{L}_+ $:
\begin{equation}\label{29.2}
\hat{f}\hat{L}_+-\hat{L}_+\hat{f}=0.
\end{equation}
Let us write down the matrix element of this equation corresponding to the transition $ n, L, M \to n', L, M + 1 $. Taking into account the fact that the matrix of the quantity $ L_+ $ has only elements with $ n, L, M \to n, L, M + 1 $, we obtain
\begin{multline*}
\langle n',L,M+1|f|n,L,M+1\rangle\langle n,L,M+1|L_+|n,L,M\rangle=\\
=\langle n',L,M+1|L_+|n',L,M\rangle\langle n',L,M|f|n,L,M\rangle,
\end{multline*}
and since the matrix elements of the quantity $ L_+ $ are independent of the suffix $ n $, we find
\begin{equation}\label{29.3}
\langle n',L,M+1|f|n,L,M+1\rangle=\langle n',L,M|f|n,L,M\rangle,
\end{equation}
whence it follows that all the quantities $ \langle n', L, M| f|n, L, M\rangle $ for different $ M $ (the other suffixes being the same) are equal.

If we apply this result to the Hamiltonian itself, we obtain our previous result that the energy of the stationary states is independent of $ M $, i.e. that the energy levels have $ (2L + 1) $-fold degeneracy.

Next, let $ \bm{A} $ be some vector physical quantity characterizing a closed system. When the system of coordinates is rotated (and, in particular, in an infinitely small rotation, i.e. when the angular momentum operator is applied), the components of a vector are transformed into linear functions of one another. Hence, as a result of the commutation of the operators $\hat{L}_i$ with the operators $\hat{A}_i$, we must again obtain components of the same vector, $\hat{A}_i$. The exact form can be found by noticing that, in the particular case where $ \bm{A} $ is the radius vector of the particle, the formulae \eqref{26.4} must be obtained. Thus we find the commutation rules
\begin{equation}\label{29.4}
\{ \hat{L}_i,\hat{A}_k \}=\i e_{ikl}\hat{A}_l
\end{equation}


These relations enable us to obtain several results concerning the form of the matrices of the components of the vector $ \bm{A} $ (M. Born, W. Heisenberg and P. Jordan 1926). First of all, it is possible to derive selection rules which determine the transitions for which the matrix elements can be different from zero. We shall not go through the fairly lengthy calculations here, however, since it will appear later (\S107) that these rules are actually a direct consequence of the general transformation properties of vector quantities and can be derived from the latter with hardly any calculation at all. Here we shall merely give the rules, without proof.

The matrix elements of all the components of a vector can be different from zero only for transitions in which the angular momentum $ L $ changes by not more than one unit:
\begin{equation}\label{29.5}
L\to L,L+1
\end{equation}
There is a further selection rule which forbids transitions between any two states with $ L = 0 $. This rule is an obvious consequence of the complete spherical symmetry of states with angular momentum zero.

The selection rules for the angular momentum component $ M $ are different for the different components of a vector: the matrix elements can be different from zero for transitions where $ M $ changes as follows:
\begin{equation}\label{29.6}
\begin{split}
M\to M+1\quad\text{for}\quad &A_+=A_x+\i A_y,\\
M\to M-1\quad\text{for}\quad &A_-=A_z-\i A_y,\\
M\to M\quad\quad\text{for}\quad&A_z.
\end{split}
\end{equation}


Moreover, it is possible to determine a general form for the matrix elements of a vector as functions of the number $ M $. These important and frequently used formulae are given here, also without proof, since they are actually a particular case of more general relations derived in \S107 for any tensor quantities.

The non-zero matrix elements of the quantity $ A_z $ are given by the formulae
\begin{equation}\label{29.7}
\begin{split}
\langle n'LM|A_z|nLM\rangle&=\frac{M}{\sqrt{L(L+1)(2L+1)}}\langle n'L||A||nL\rangle,\\
\langle n'LM|A_z|n,L-1,M\rangle&=\sqrt{\frac{L^2-M^2}{L(2L-1)(2L+1)}}\langle n'L||A||n,L-1\rangle,\\
\langle n',L-1,M|A_z|nLM\rangle&=\sqrt{\frac{L^2-M^2}{L(2L-1)(2L+1)}}\langle n',L-1||A||nL\rangle.
\end{split}
\end{equation}
Here the symbol \[ \langle
 n'L'||A||nL\rangle \] 
denotes a \textit{reduced matrix element}, a quantity independent of the quantum number $ M $.\footnote{The appearance in formulae \eqref{29.7} and \eqref{29.9} of denominators which depend on $ L $ is in accordance with the general notation used in \S107. The convenience of these denominators is shown, in particular, by the simple form of equation \eqref{29.12} for the matrix elements of the scalar product of two vectors.
	

The symbol for the reduced matrix element is to be taken as a whole, in contrast to the matrix element symbol (see the comments following \eqref{11.17}).
} These matrix elements are related by
\begin{equation}\label{29.8}
\langle
n'L'||A||nL\rangle=\langle
nL||A||n'L'\rangle^*,
\end{equation}
which follows directly from the fact that the operator is $\hat{A}_z$ Hermitian.

The matrix elements of the quantities $ A_- $ and $ A_+ $ are also determined by the reduced matrix elements. The non-zero matrix elements of $ A_- $ are.
\begin{multline}\label{29.9}
\langle n',L,M-1|A_-|nLM\rangle=\sqrt{\frac{(L-M+1)(L+M)}{L(L+1)(2L+1)}}\langle n'L||A||nL\rangle,\\
\langle n',L,M-1|A_-|n,L-1,M\rangle=\sqrt{\frac{(L-M+1)(L-M)}{L(2L-1)(2L+1)}}\langle n'L||A||n,L-1\rangle,\\
\langle n',L-1,M-1|A_-|n,L,M\rangle=-\sqrt{\frac{(L+M-1)(L+M)}{L(2L-1)(2L+1)}}\langle n',L-1||A||n,L\rangle
\end{multline}
The matrix elements of $ A_+ $ need not be written out separately: since $ A_x $ and $ A_y $ are real we have
\begin{equation}\label{29.10}
\langle n'L'M'|A_+|nLM\rangle=\langle nLM|A_-|n'L'M'\rangle^*.
\end{equation}



There is a formula which expresses the matrix elements of the scalar $ \bm{A}\bm{B} $ in terms of the reduced matrix elements of the two vector quantities $ \bm{A} $ and $ \bm{B} $. The calculation is conveniently carried out by writing the operator $ \hat{\bm{A}}\hat{\bm{B}} $ in the form
\begin{equation}\label{29.11}
\hat{\bm{A}}\hat{\bm{B}}=\frac{1}{2}(\hat{A}_+\hat{B}_-+\hat{A}_-\hat{B}_+)+\hat{A}_z\hat{B}_z.
\end{equation}
The matrix of $ \bm{A}\bm{B} $ (like that of any scalar) is diagonal with respect to $ L $ and $ M $. A calculation by means of formulae \eqref{29.7}–\eqref{29.9} gives the result
\begin{equation}\label{29.12}
\langle n'LM|\bm{A}\bm{B}|nLM\rangle=\frac{1}{2L+1}\sum_{n'',L''}\langle n'L||A||n''L''\rangle\langle n''L''||B||nL\rangle,
\end{equation}
where $ L'' $ takes the values $ L, L\pm 1 $.

For reference, we shall give the reduced matrix elements for the vector $ \bm{L} $ itself. A comparison of \eqref{29.9} and \eqref{27.12} shows that
\begin{equation}\label{29.13}
\begin{split}
\langle L||L|L\rangle=\sqrt{L(L+1)(2L+1)},\\
\langle L-1||L||L\rangle=\langle L||L||L-1\rangle=0.
\end{split}
\end{equation}


A quantity that often occurs in applications is the unit vector $ \bm{n} $ along the radius vector of the particle. Its reduced matrix elements can be calculated by finding, for example, the matrix elements of $ n_z = \cos\theta $ for a zero angular-momentum component, $ m = 0 $;
\[ \langle l-1,0|n_z|l0\rangle=\int_{0}^{\pi}\Theta_{l-1,0}^*\cos\theta\cdot\Theta_{l0}\sin\theta\d\theta \]
with the functions $\Theta_{l0}$ given by \eqref{28.11}. The evaluation of the integral\footnote{By $ l-1 $ times integrating by parts with $ \d \cos\theta $; the general formula for integrals of this type is (107.14).} gives
\[ \langle l-1,0|n_z|l0\rangle=\frac{\i l}{\sqrt{(2l-1)(2l+1)}}. \]
The matrix elements for transitions $ l \to l $ are zero (as for any polar vector of an individual particle; see (30.8) below). Comparison with \eqref{29.7} then gives
\begin{equation}\label{29.14}
\langle l-1||n||l\rangle=-\langle l||n||l-1\rangle=\i\sqrt{l},\quad\langle l||n||l\rangle=0.
\end{equation}






{\small
\textbf{PROBLEM}


Average the tensor $ n_in_k-(1/3)\delta_{ik} $ (where $ \bm{n} $ is a unit vector along the radius vector of a particle) over a state where the magnitude but not the direction of the vector $ \bm{1} $ is given (i.e. $ l_z $ is indeterminate).





SOLUTION. The required mean value is an operator which can be expressed in terms of the operator $\hat{\bm{l}}$ alone. We seek it in the form
\[ \bar{n_in_k}-\frac{1}{3}\delta_{ik}=a\left[\hat{l}_i\hat{l}_k+\hat{l}_k\hat{l}_i-\frac{2}{3}\delta_{ik}l(l+1) \right]; \]
this is the most general symmetrical tensor of rank two with zero trace that can be formed from the components of $\hat{\bm{l}}$. To determine the constant a we multiply this equation on the left by $ \hat{l}_i $ and on the right by $\hat{l}_k$ (summing over $ i $ and $ k $). Since the vector $ \bm{n} $ is perpendicular to the vector $ \h\hat{\bm{l}}=\hat{\bm{r}
}\times\hat{\bm{p}} $, we have $ n_i\hat{l}_i=0 $. The product $ \hat{l}_i\hat{l}_i\hat{l}_k\hat{l}_k=(\hat{\bm{l}}^2)^2 $ is replaced by its eigenvalue $ l^2(l + 1)^2 $, and the product is transformed by means of the commutation relations \eqref{26.7} as follows:
\begin{multline*}
\hat{l}_i\hat{l}_k\hat{l}_i\hat{l}_k=\hat{l}_i\hat{l}_i\hat{l}_k\hat{l}_k-\i e_{ikl}\hat{l}_i\hat{l}_l\hat{l}_k=(\hat{\bm{l}})^2-\frac{\i}{2}e_{ikl}\hat{l}_i(\hat{l}_l\hat{l}_k-\hat{l}_k\hat{l}_l)=\\
=(\hat{\bm{l}})^2+\frac{1}{2}e_{ikl}e_{lkm}\hat{l}_i\hat{l}_m=(\hat{\bm{l}})^2-\hat{\bm{l}}^2=l^2(l+1)^2-l(l+1)
\end{multline*}
(using the fact that $ e_{ikl}e_{mkl} = 2\delta_{im} $). After a simple reduction we obtain the result
\[ a=-\frac{1}{(2l-1)(2l+3)}. \]}





\section{Parity of a state}\label{Parity of a state}


Besides the parallel displacements and rotations of the coordinate system, the invariance under which represents the homogeneity and isotropy of space respectively, there is another transformation which leaves unaltered the Hamiltonian of a closed system. This is what is called the \textit{inversion transformation}, which consists in simultaneously changing the sign of all the coordinates, i.e. a reversal of the direction of each coordinate axis; a right-handed coordinate system then becomes left-handed, and vice versa. The invariance of the Hamiltonian under this transformation expresses the symmetry of space under mirror reflections.\footnote{Invariance under inversion exists also for the Hamiltonian of a system of particles in a centrally symmetric field with the centre at the origin.
} In classical mechanics, the invariance of Hamilton’s function with respect to inversion does not lead to a conservation law, but the situation is different in quantum mechanics.

Let us denote by $\hat{P}$ (for “parity”) an inversion operator whose effect on a wave function $\psi(\bm{r})$ is to change the sign of the coordinates:
\begin{equation}\label{30.1}
\hat{P}\psi(\bm{r})=\psi(-\bm{r})
\end{equation}
It is easy to find the eigenvalues $ P $ of this operator, which are determined by the equation
\begin{equation}\label{30.2}
\hat{P}\psi(\bm{r})=P\psi(\bm{r})
\end{equation}
To do so, we notice that a double application of the inversion operator amounts to identity: the argument of the function is unchanged. In other words, we have $ \hat{P}^2\psi=P^2\psi=\psi $, i.e. $ P^2 = 1 $, whence
\begin{equation}\label{30.3}
P=\pm 1
\end{equation}
Thus the eigenfunctions of the inversion operator are either unchanged or change in sign when acted upon by this operator. In the first case, the wave function (and the corresponding state) is said to be \textit{even}, and in the second it is said to be \textit{odd}.

The invariance of the Hamiltonian under inversion (i.e. the fact that the operators $\hat{H}$ and $ \hat{P} $ commute) thus expresses the \textit{law of conservation of parity}: if the state of a closed system has a definite parity (i.e. if it is even, or odd) then this parity is conserved in the course of time.\footnote{To avoid misunderstanding, it should be mentioned that this refers to the non-relativistic theory. There exist interactions in Nature, falling in the realm of relativistic theory, which violate the conservation of parity.
}

The angular momentum operator also is invariant under inversion, which changes the sign of the coordinates and of the operators of differentiation with respect to them; the operator \eqref{26.2} thus remains unaltered. In other words, the inversion operator commutes with the angular momentum operator, and this means that the system can have a definite parity simultaneously with definite values of the angular momentum $ L $ and its component $ M $. All states that differ only in the value of $ M $ have the same parity; this is evident because the properties of a closed system are independent of its orientation in space, and it can be formally demonstrated from the commutation rule $ \hat{L}_+\hat{P}-\hat{P}\hat{L}_+ $ by the same method as in deriving \eqref{29.3} from \eqref{29.2}.

There are specific \textit{parity selection rules} for the matrix elements of various physical quantities. Let us first consider scalars. Here we must distinguish \textit{true scalars}, which are unchanged by inversion, from \textit{pseudoscalars}, which change sign, for instance the scalar product of an axial and a polar vector The operator of a true scalar $ f $ commutes with $\hat{P}$; hence it follows that, if the matrix of $ P $ is diagonal, then the matrix of $ f $ is diagonal also as regards the parity suffix, i.e. the matrix elements are zero except for transitions $ g \to g $ and $ u \to u $ (where $ g $ and $ u $ denote even and odd states respectively). For the operator of a pseudoscalar quantity, we have $ \hat{P}\hat{f}=-\hat{f}\hat{P} $; the operators $ \hat{P} $ and $\hat{f}$ \textit{anticommute}. The matrix element of this equation for a transition $ g \to g $ is \[ P_{gg}f_{gg} = - f_{gg}P_{gg}, \] and so $ f_{gg} = 0 $ since $ P_{gg} = 1 $. Similarly we find that $ f_{uu} = 0 $. Thus, in the matrix of a pseudoscalar quantity, only those elements can be different from zero which correspond to transitions with change of parity. The selection rules for the matrix elements of scalars are therefore:
\begin{equation}\label{30.4}
\begin{split}
\text{true scalars}\qquad g\to g,u\to u;\\
\text{pseudoscalars}\qquad g\to u,u\to g.
\end{split}
\end{equation}



These rules can also be obtained directly from the definition of the matrix elements. Let us consider, for example, the integral $ f_{ug}=\int\psi_u^*\hat\psi_g\d q $, where the function $\psi_g$ is even and $\psi_u$ odd. When all the coordinates change sign, the integrand does so if $ f $ is a true scalar; on the other hand, the integral taken over all space cannot change when the variables of integration are renamed. Hence it follows that $ f_{ug} = - f_{ug} $, i.e. $ f_{ug} \equiv 0 $.

We can similarly derive selection rules for vector quantities. Here it must be recalled that ordinary (polar) vectors change sign on inversion, while axial vectors (such as the angular momentum vector, which is the vector product of the two polar vectors $ \bm{p} $ and $ \bm{r} $) are unchanged by inversion. The selection rules are found to be:
\begin{equation}\label{30.5}
\begin{split}
\text{polar vectors}\qquad g\to u,u\to g;\\
\text{axial vectors}\qquad g\to g,u\to u.
\end{split}
\end{equation}
Let us determine the parity of the state of a single particle with angular momentum $ l $. The inversion transformation ($ x \to-x, y \to-y, z \to-z $) is, in spherical polar coordinates, the transformation
\begin{equation}\label{30.6}
r\to r,\quad\theta\to\pi-\theta,\quad\phi\to\phi+\pi.
\end{equation}
The dependence of the wave function of the particle on the angle is given by the spherical harmonic $ Y_{lm} $, which, apart from a constant that is here unimportant, has the form $ P_l^m(\cos\theta)\e^{\i m\phi} $. When $\phi$ is replaced by $ \phi+\pi $, the factor $ \e^{\i m\phi} $ is multiplied by $ (-1)^m $, and when $\theta$ is replaced by $ \pi-\theta $, $ P_l^m(\cos\theta) $ becomes $ P_l^m(-\cos\theta)=(-1)^{l-m}P_l^m(\cos\theta) $. Thus the whole function is multiplied by $ (-1)^l $ (independent of $ m $, in agreement with what was said above), i.e. the parity of a state with a given value of $ l $ is
\begin{equation}\label{30.7}
P=(-1)^l.
\end{equation}
We see that all states with even $ l $ are even, and all those with odd $ l $ are odd.

A vector physical quantity relating to an individual particle can have non-zero matrix elements only for transitions with $ l \to l $ or $ l \pm 1 $ (\S\ref{Matrix elements of vectors}). Remembering this, and comparing formula \eqref{30.7} with what was said above regarding the change of parity in the matrix elements of vectors, we reach the result that the matrix elements of vectors relating to an individual particle are zero except for the transitions:
\begin{equation}\label{30.8}
\begin{split}
\text{polar vectors}\qquad l&\to l\pm 1,\\
\text{axial vectors}\qquad l&\to 1.
\end{split}
\end{equation}






\section{Addition of angular momenta}\label{Addition of angular momenta}


Let us consider a system composed of two parts whose interaction is weak. If the interaction is entirely neglected, then for each part the law of conservation of angular momentum holds. The angular momentum $ \bm{L} $ of the whole system can be regarded as the sum of the angular momenta $ \bm{L}_1 $ and $ \bm{L}_2 $ of its parts. In the next approximation, when the weak interaction is taken into account, $ \bm{L}_1 $ and $ \bm{L}_2 $ are not exactly conserved, but the numbers $ L_1 $ and $ L_2 $ which determine their squares remain “good” quantum numbers suitable for an approximate description of the state of the system. Regarding the angular momenta in a classical manner, we can say that in this approximation $ \bm{L}_1 $ and $ \bm{L}_2 $ rotate round the direction of L while remaining unchanged in magnitude.

For such systems the question arises regarding the “law of addition” of angular momenta: what are the possible values of $ L $ for given values of $ L_1 $ and $ L_2 $? The law of addition for the components of angular momentum is evident: since $ \hat{L}_z=\hat{L}_{1z}+\hat{L}_{2z} $, it follows that
\begin{equation}\label{31.1}
M=M_1+M_2
\end{equation}
There is no such simple relation for the operators of the squared angular momenta, however, and to derive their “law of addition” we reason as follows.

If we take the quantities $ \bm{L}_1^2 ,  \bm{L}_2^2 ,  L_{1z} ,  L_{2z} $ as a complete set of physical quantities,\footnote{Together with such other quantities as form a complete set when combined with these four. These other quantities play no part in the subsequent discussion, and for brevity we shall ignore them entirely, and conventionally call the above four quantities a complete set.
} every state will be determined by the values of the numbers $ L_1, L_2, M_1, M_2 $. For given $ L_1 $ and $ L_2 $, the numbers $ M_1 $ and $ M_2 $ take $ (2L_1+1) $ and $ (2L_2 + 1) $ different values respectively, so that there are altogether $ (2L_1 + 1) (2L_2+1) $ different states with the same $ L_1 $ and $ L_2 $. We denote the wave functions of the states for this representation by $\phi_{L_1L_2M_1M_2}$.

Instead of the above four quantities, we can take the four quantities $ \bm{L}_1^2, \bm{L}_2^2, \bm{L}^2, L_z $ as a complete set. Then every state is characterized by the values of the numbers $ L_1, L_2, L, M $ (we denote the corresponding wave functions by $\psi_{L_1L_2M_1M_2}$). For given $ L_1 $ and $ L_2 $, there must of course be $ (2L_1 + 1)(2L_2 + 1) $ different states as before, i.e. for given $ L_1 $ and $ L_2 $ the pair of numbers $ L $ and $ M $ must take $ (2L_1 + 1)(2L_2 + 1) $ pairs of values. These values can be determined as follows.

By adding the various possible values of $ M_1 $ and $ M_2 $, we get the corresponding values of $ M $, as shown below:

\[ \begin{tabularx}{0.5\textwidth}{XXX}
$ M_1 $ & $ M_2 $ & $ M $\\
\hline
$ L_1 $ & $ L_2 $ & $ L_1+L_2 $\\
$ L_1 $ & $ L_2-1 $ & $ L_1+L_2-1 $\\
$ L_1-1 $ & $ L_2 $ & $ L_1+L_2-1 $\\
$ L_1-1 $ & $ L_2-1 $ & $ L_1+L_2-2 $\\
$ L_1 $ & $ L_2-2 $ & $ L_1+L_2-2 $\\
$ L_1-2 $ & $ L_2 $ & $ L_1+L_2-2 $\\
\dots &\dots&\dots\\
\end{tabularx}
 \]




We see that the greatest possible value of $ M $ is $ M = L_1 + L_2 $, corresponding to one state $\phi$ (one pair of values of $ M_1 $ and $ M_2 $). The greatest possible value of $ M $ in the states $\psi$, and hence the greatest possible value of $ L $ also, is therefore $ L_1 + L_2 $. Next, there are two states $\phi$ with $ M = L_1 + L_2 - 1 $. Consequently, there must also be two states $\psi$ with this value of $ M $; one of them is the state with $ L = L_1 + L_2 $ (and $ M = L - 1 $), and the other is that with $ L = L_1 + L_2 - 1 $ (and $ M = L $). For the value $ M = L_1 + L_2-2 $ there are three different states $\phi$. This means that, besides the values $ L = L_1 + L_2, L = L_1 + L_2 - 1 $, the value $ L = L_1 + L_2 - 2 $ can occur.

The argument can be continued in this way so long as a decrease of $ M $ by $ 1 $ increases by $ 1 $ the number of states with a given $ M $. It is easily seen that this is so until $ M $ reaches the value $ |L_1 - L_2| $. When $ M $ decreases further, the number of states no longer increases, remaining equal to $ 2L_2 + 1$ (if $ L_2 \leqslant L_1 $) . Thus $ |L_1 - L_2| $ is the least possible value of $ L $, and we arrive at the result that, for given $ L_1 $ and $ L_2 $, the number $ L $ can take the values
\begin{equation}\label{31.2}
L=L_1+L_2,L_1+L_2-1,\dots,|L_1-L_2|,
\end{equation}
that is $ 2L_2 + 1 $ different values altogether (supposing that $ L_2 \leqslant L_1 $). It is easy to verify that we do in fact obtain $ (2L_1 + 1)(2L_2+1) $ different values of the pair of numbers $ M $, $ L $. Here it is important to note that, if we ignore the $ 2L + 1 $ values of $ M $ for a given $ \hat{L} $, then only one state will correspond to each of the possible values \eqref{31.2} of $ L $.

This result can be illustrated by means of what is called the \textit{vector model}. If we take two vectors $ \bm{L}_1, \bm{L}_2 $ of lengths $ L_1 $ and $ L_2 $, then the values of $ L $ are represented by the integral lengths of the vectors $ \bm{L} $ which are obtained by vector addition of $ \bm{L}_1 $ and $ \bm{L}_2 $; the greatest value of $ L $ is $ L_1 + L_2 $, which is obtained when $ \bm{L}_1 $ and $ \bm{L}_2 $ are parallel, and the least value is $ |L_1 - L_2| $, when $ \bm{L}_1 $ and $ \bm{L}_2 $ are antiparallel.

In states with definite values of the angular momenta $ L_1 $, $ L_2 $ and of the total angular momentum $ L $, the scalar products $ \bm{L}_1\bm{L}_2 $, $ \bm{L}\bm{L}_1 $ and $ \bm{L}\bm{L}_2 $ also have definite values. These values are easily found. To calculate $ \bm{L}_1\bm{L}_2 $, we write $ \hat{\bm{L}}=\hat{\bm{L}}_1+\hat{\bm{L}}_2 $ or, squaring and transposing,
\[ 2\hat{\bm{L}}_1\hat{\bm{L}}_2=\hat{\bm{L}}^2-\hat{\bm{L}}_1^2-\hat{\bm{L}}_2^2. \]
Replacing the operators on the right-hand side of this equation by their eigenvalues, we obtain the eigenvalue of the operator on the left-hand side:
\begin{equation}\label{31.3}
\bm{L}_1\bm{L}_2=\frac{1}{2}\{ L(L+1)-L_1(L_1+1)-L_2(L_2+1) \}.
\end{equation}
Similarly we find
\begin{equation}\label{31.4}
\bm{L}\bm{L}_1=\frac{1}{2}\{L(L+1)+L_1(L_1+1)-L_2(L_2+1) \}
\end{equation}



Let us now determine the “addition rule for parities”. The wave function $\Psi$ of a system consisting of two independent parts is the product of the wave functions $\psi_1$ and $\Psi_2$ of these parts. Hence it is clear that if the latter are of the same parity (i.e. both change sign, or both do not change sign, when the sign of all the coordinates is reversed), then the wave function of the whole system is even. On the other hand, if $\Psi_1$ and $\Psi_2$ are of opposite parity, then the function $\Psi$ is odd. These statements may be written
\begin{equation}\label{31.5}
P=P_1P_2
\end{equation}
where $ P $ is the parity of the whole system and $ P_1 $, $ P_2 $ those of its parts. This rule can, of course, be generalized at once to the case of a system composed of any number of non-interacting parts.

In particular, if we are concerned with a system of particles in a centrally symmetric field (the mutual interaction of the particles being supposed weak), then the parity of the state of the whole system is given by
\begin{equation}\label{31.6}
P=(-1)^{l_1+l_2+\dots}
\end{equation}
see \eqref{30.7}. We emphasize that the exponent here contains the algebraic sum of the angular momenta $ l_i $, and this is not in general the same as their “vector sum”, i.e. the angular momentum $ L $ of the system.

If a closed system disintegrates (under the action of internal forces), the total angular momentum and parity must be conserved. This circumstance may render it impossible for a system to disintegrate, even if this is energetically possible.

For instance, let us consider an atom in an even state with angular momentum $ L = 0 $, which is able, so far as energy considerations go, to disintegrate into a free electron and an ion in an odd state with the same angular momentum $ L = 0 $. It is easy to see that in fact no such disintegration can occur (it is, as we say, \textit{forbidden}). For, by virtue of the law of conservation of angular momentum, the free electron would also have to have zero angular momentum, and therefore be in an even state $ (P = (−1)^0 = + 1) $; the state of the system ion$ + $electron would then be odd, however, whereas the original state of the atom was even.
